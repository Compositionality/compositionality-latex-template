\documentclass[
    accepted=2020-05-01,
    issue=1, 
    volume=2, 
]{compositionalityarticle}

\usepackage[utf8]{inputenc}
\usepackage[english]{babel}
\usepackage[T1]{fontenc}

\usepackage{tikz}
\usepackage{lipsum}

% commands che danno errore
\newenvironment{widetext}{}{}
% \let\affiliation\relax
% \let\email\relax
% \let\orcid\relax
% \let\homepage\relax  
\newtheorem{theorem}{Theorem}

\begin{document}

\title{Template demonstrating the compositionalityarticle class}
\date{}
\author{Brendan Fong}
\email{latex@compositionality-journal.org}
\homepage{http://compositionality-journal.org}
\orcid{0000-0003-0290-4698}
\thanks{You can use the \texttt{\string\email}, \texttt{\string\homepage}, \texttt{\string\orcid}, and \texttt{\string\thanks} commands to add additional information for the preceding \texttt{\string\author}. If applicable, this can also be used to indicate that a work has previously been published in conference proceedings.}
\affiliation{MIT, Mathematics Department, Boston, USA}
\author{Nina Otter}
\email{latex@compositionality-journal.org}
\orcid{0000-0002-3578-1420}
\affiliation{UCLA, Mathematics Department, Los Angeles, USA}
\author{Joshua Tan}
\email{latex@compositionality-journal.org}
\orcid{0000-0001-5317-4782}
\affiliation{University of Oxford, Computer Science Department,
  Oxford, UK}

% \maketitle
\printtitlestuff

\begin{abstract}
  In the standard \texttt{onecolumn} layout
 the abstract is typeset above the text.
  Compositionality also supports a \texttt{twocolumn} layout with the abstract as a boldface first paragraph.
  Both can be combined with the \texttt{titlepage} option to obtain a format with dedicated title and abstract pages that are not included in the page count.
  This format can be more suitable for long articles.
  The \texttt{abstract} environment can appear both before and after the \texttt{\string\maketitle} command and calling \texttt{\string\maketitle} is optional, as long as there is an \texttt{abstract}.
  Both \texttt{abstract} and \texttt{\string\maketitle} however must be placed after all other \texttt{\string\author}, \texttt{\string\affiliation}, etc.\ commands.
In the \texttt{twocolumn} layout and without the \texttt{titlepage} option there can be a paragraph directly following the abstract before the first section.
In \texttt{onecolumn} format or with a dedicated \texttt{titlepage}, this should be avoided.
Note that clicking the title links to the page for the published article on \href{http://compositionality-journal.org}{compositionality-journal.org}.
In this way readers can easily verify whether a work using the \texttt{compositionalityarticle} class was actually published in Compositionality.
\end{abstract}



\section{Figures}
\begin{figure}[t]
  \centering
 \includegraphics[scale=.2]{example-image-a}
  \caption{Every figure must have an informative caption and a number.
    The caption can be placed above, below, or to the side of the figure, as you see fit.
    The same applies for tables, boxes, and other floating elements.}
  \label{fig:figure1}
\end{figure}
See Fig.~\ref{fig:figure1} for an example of how to include figures.
Feel free to place them at the top or bottom of the page, or in the middle of a paragraph as you see fit.
Try to place them on the same page as the text referring to them.
A figure on the first page can help readers remember and recognize your work more easily.

\section{Sectioning and equations}
Sections, subsections, subsubsections, and paragraphs should be typeset with the standard LaTeX commands.
You can use the standard commands for equations.
For multi-line equations \texttt{align} is \href{http://tex.stackexchange.com/questions/196/eqnarray-vs-align}{preferable} over \texttt{eqnarray}.
\begin{align}
  \label{emc}
  E &= m c^2\\
  a^2 + b^2 &= c^2\\
  H |\psi\rangle &= E |\psi\rangle\\
  (\mathbb{I} \otimes A) (B \otimes \mathbb{I}) &= A \otimes B
\end{align}
Whether you prefer to refer to equations as Eq.~\eqref{emc}, Equation~\ref{emc}, or just \eqref{emc} is up to you, but please be consistent and use the \texttt{\string\eqref\{\dots\}} command instead of writing \texttt{(\string\ref\{\dots\})}.
As a courtesy for your readers and referees please suppress equation numbers only if there is a valid reason to do so to not make in unnecessarily difficult to refer to individual results and steps in derivations. 

\paragraph{Paragraphs}
The paragraph is the smallest unit of sectioning.
Feel free to end the paragraph title with a full stop if you find this appropriate.

\subsection{References and footnotes}
\label{sec:subsec1}
Footnotes\footnote{Only use footnotes when appropriate.} appear in the bottom of the page.
Please do not mix them with your references.
Citations to other works should appear in the References section at the end of the work.

\begin{theorem}[DOI links are required]
  Important: As Compositionality is a member of Crossref, all references to works that have a DOI must be hyperlinked according to the DOI. Those links must start with \texttt{https://doi.org/}. Direct links to the website of the publisher are not sufficient.
\end{theorem}

This can be achieved in several ways, depending on how you format your bibliography.
Suppose the DOI of an article \cite{examplecitation} that you want to cite is \texttt{10.22331/idonotexist}.
If you are formating your bibliography manually, you can cite this work using the following in your \texttt{thebibliography} environment:
\begin{verbatim}
\bibitem{examplecitation}
  Name Surname,
  Compositionality \textbf{123}, 123456 (1916).
  \href{https://doi.org/10.22331/idonotexist}
       {https://doi.org/10.22331/idonotexist}
\end{verbatim}

\begin{theorem}[One citation per bibitem]
  Important: If you are formatting your bibliography manually, please do not group multiple citations into one \texttt{\string\bibitem}.
  It is not nice for your readers and it screws up our automatic extraction of citations and creates superfluous manual work.
\end{theorem}

We encourage the use of BibTeX to generate your bibliography from the BibTeX meta-data provided by publishers.
For DOI linking to work, the BibTeX file must contain the \texttt{doi} field as for example in:
\begin{verbatim}
@article{examplecitation,
  author = {Surname, Name},
  title = {Title},
  journal = {Compositionality},
  volume = {123},
  page = {123456},
  year = {1916},
  doi = {10.22331/idonotexist},
}
\end{verbatim}

Be advised that \href{http://wiki.lyx.org/BibTeX/Tips}{BibTeX does not support unicode characters}.
All special characters must be input via their respective LaTeX commands.

If you are using BibTeX, you can load the \texttt{natbib} package by putting
\begin{verbatim}
\usepackage[numbers,sort&compress]{natbib}
\end{verbatim}
in the preamble of your document and then use the \texttt{plainnat} citation style by including your BibTeX bibliography \texttt{mybibliography.bib} where you want the bibliography to appear as follows:
\begin{verbatim}
\bibliographystyle{plainnat}
\bibliography{mybibliography}
\end{verbatim}
The \texttt{compositionalityarticle} class automatically detects that the \texttt{natbib} package was loaded and redefines the \texttt{\string\doi} command to create hyperlinks.
This is likely the easiest option if you are converting from another document class.

If you want to used BibLaTeX, you can instead add
\begin{verbatim}
\usepackage[backend=bibtex]{biblatex}
\addbibresource{mybibliography.bib}
\end{verbatim}
to the preamble of your document and then output the bibliography with
\begin{verbatim}
\printbibliography
\end{verbatim}
where appropriate.
You then have to upload the .bbl file along with the other source files when submitting to the arXiv.
Due to incompatibilities between different LaTeX versions we unfortunately cannot recommend this option until the arXiv has updated its LaTeX system \cite{biblatexsubmittingtothearxiv}.

The \texttt{compositionalityarticle} class automatically detects that the \texttt{biblatex} package was loaded, sets the default option \texttt{doi=true} to include the DOI in the bibliography, and declares a suitable field format to make it a hyperlink.
Due to issues with \texttt{biber} we recommend to use the \texttt{bibtex} backend of \texttt{biblatex}.

More information on how to get DOI links in your document can be found on StackExchange \cite{howtogetdoilinksinbibliography,automaticallyaddingdoifieldstoahandmadebibliography}.
Feel free to change the appearance of citations in any way you like by using a different \texttt{bibliographystyle} or via the advanced mechanisms provided by BibLaTeX.
The only two requirements are that citations must uniquely identify the cited work and that they must contain a DOI hyperlink whenever possible.

To get correct line breaks within links, please add the line 
\begin{verbatim}
\pdfoutput=1
\end{verbatim}
within the first five lines of your main LaTeX file as suggested by the arXiv \cite{arxivpdfoutput}.

\section{Unpublished and accepted works}
This class supports the \texttt{unpublished} option which disables all Compositionality related branding such as the color as well as the hyperlink in the title.
It can be used for works not yet accepted in Compositionality or  that are not intended for submission to Compositionality.
By giving the \texttt{accepted=YYYY-MM-DD} option, with \texttt{YYYY-MM-DD} the acceptance date, the footnote ``Accepted in Compositionality on YYYY-MM-DD. Click on title to verify.'' is  added to the bottom of each page to clearly mark works that have been accepted in Compositionality. Once the work is accepted for publication you will receive an identifier {\tt article-id} which you will have to give as the {\tt shortitle=article-id} option (this is needed for the title hyperlink). Finally, for accepted works you also have to  give the  \texttt{volume=volume-nr}, as well as \texttt{issue=issue-nr} options.

\section{Summary section}
Longer articles should include a section that, early on, explains the main results, their limitations, and assumptions.
Depending on the target audience this section can be used to, for example, present the main theorem, or provide a summary of the results for a wider audience.

\section{Extra packages}
Compositionality encourages you to load the following extra packages:
\begin{verbatim}
\usepackage[utf8]{inputenc}
\usepackage[english]{babel}
\usepackage[T1]{fontenc}
\usepackage{amsmath}
\usepackage{hyperref}
\end{verbatim}
If you do not load the \texttt{hyperref} package, \texttt{compositionalityarticle} automatically loads it for you.
Packages that change font settings, such as \texttt{times} or \texttt{helvet} should be avoided.

\section{Wide equations}
Very wide equations can be shown expanding over both columns using the \texttt{widetext} environment.
In \texttt{onecolumn} mode, the \texttt{widetext} environment has no effect.
\begin{widetext}
  \begin{equation}
|\mathrm{AME}(n=6,q=5)\rangle=\sum_{i,j,k=0}^4 |i,j,k,i+j+k,i+2j+3k,i+3j+4k\rangle
  \end{equation}
\end{widetext}
To enabled this feature in the \texttt{twocolumn} mode, \texttt{compositionalityarticle} relies on the package \texttt{ltxgrid}.
Unfortunately this package has a bug that leads to a sub-optimal placement of extremely long footnotes.

\section{Title information}
You can provide information on authors and affiliations in the common format also used by \texttt{revtex}:
\begin{verbatim}
\title{Title}
\author{Author 1}
\author{Author 2}
\affiliation{Affiliation 1}
\author{Author 3}
\affiliation{Affiliation 2}
\author{Author 4}
\affiliation{Affiliation 1}
\affiliation{Affiliation 3}
\end{verbatim}
In this example affiliation 1 will be associated with authors 1, 2, and 4, affiliation 2 with author 3 and affiliation 3 with author 4.
Repeated affiliations are automatically recognized and typeset in the \texttt{superscriptaddress} style.
Alternatively you can use a format similar to that of the \texttt{authblk} package and the \texttt{elsearticle} document class to specify the same affiliation relations as follows:
\begin{verbatim}
\title{Title}
\author[1]{Author 1}
\author[1]{Author 2}
\author[2]{Author 3}
\author[1,3]{Author 4}
\affil[1]{Affiliation 1}
\affil[2]{Affiliation 1}
\affil[3]{Affiliation 1}
\end{verbatim}






\appendix

\section{First section of the appendix}
Compositionality allows the usage of appendices.

\subsection{Subsection}
Ideally, the command \texttt{\string\appendix} should be put before the appendices to get appropriate section numbering.
The appendices are then numbered alphabetically, with numeric (sub)subsection numbering.
Equations continue to be numbered sequentially.
\begin{equation}
  A \neq B
\end{equation}
You are free to change this in case it is more appropriate for your article, but a consistent and unambiguous numbering of sections and equations must be ensured.

If you want your appendices to appear in \texttt{onecolumn} mode but the rest of the document in \texttt{twocolumn} mode, you can insert the command \texttt{\string\onecolumn\string\newpage} before \texttt{\string\appendix}.   

\section{Problems and Bugs}
In case you encounter problems using the \texttt{compositionalityarticle} class please analyse the error message carefully and look for help online; \href{http://tex.stackexchange.com/}{http://tex.stackexchange.com/} is an excellent resource.
If you cannot resolve a problem,  open a bug report in our bug-tracker under 
\begin{center}
  \href{https://github.com/compositionality/issues}{https://github.com/compositionality/latex-template/issues}.
\end{center}

You can also contact us via email under \href{executive-board@compositionality-journal.org}{executive@compositionality-journal.org}, but it may take significantly longer to get a response.
In any case we need the full source of a document that produces the problem and the log file showing the error to help you.

\bibliographystyle{plain}
\begin{thebibliography}{9}
\bibitem{examplecitation}
  Name Surname,
  Compositionality \textbf{123}, 123456 (1916).
  \href{https://doi.org/10.22331/idonotexist}
       {https://doi.org/10.22331/idonotexist}

\bibitem{biblatexsubmittingtothearxiv}
  StackExchange discussion on \href{http://tex.stackexchange.com/questions/26990/biblatex-submitting-to-the-arxiv}{``Biblatex: submitting to the arXiv'' (2017-01-10)}

\bibitem{arxivpdfoutput}
  Help article published by the arXiv on \href{https://arxiv.org/help/submit_tex}{``Considerations for TeX Submissions'' (2017-01-10)}

\bibitem{howtogetdoilinksinbibliography}
  StackExchange discussion on \href{http://tex.stackexchange.com/questions/3802/how-to-get-doi-links-in-bibliography}{``How to get DOI links in bibliography'' (2016-11-18)}
  
\bibitem{automaticallyaddingdoifieldstoahandmadebibliography}
  StackExchange discussion on \href{http://tex.stackexchange.com/questions/6810/automatically-adding-doi-fields-to-a-hand-made-bibliography}{``Automatically adding DOI fields to a hand-made bibliography'' (2016-11-18)}

\end{thebibliography}



\onecolumn\newpage


\end{document}